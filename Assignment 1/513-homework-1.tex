\documentclass[a4paper,12pt]{article}
\usepackage{amssymb}
\usepackage{amsfonts}
\begin{document}
\pagestyle{headings}
\markright{Design and Analysis of Algorithms}

\centerline{Eric Bronner, Aedan Dispenza, Timothy Yong, Jason Davis}
\centerline{CS-513 - Dr. Farach-Colton}
\centerline{\underline{Homework 1}}

\underline{Problem 1}\\
\indent $d+1 \leq n \leq 2^{d+1} -  1$\\
\indent Proof by Induction:\\

1. Base Case:\\
\indent \indent tree of $n =1$, $d = 0$.
\begin{center}
$0 + 1 \leq 1 \leq 2^{0+1} - 1$\\
$1 \leq 1 \leq 1$\\
\end{center}

2. Inductive Hypothesis:\\
\indent \indent Assume the number of nodes n in a completely sparse tree $n(d-1) = d$.\\

3. Inductive Step:
\begin{center}
$n(d) = n(d-1)+ 1$\\
$n(d) = d + 1$ $\square$\\
\end{center}

Proof by Induction:\\

1. Base Case:\\
\indent \indent tree of $n = 1$, $d = 0$.
\begin{center}
$\log{n} \leq d \leq n-1$ \\
$log{1} \leq 0 \leq 1-1$\\
$0 \leq 0 \leq 0$\\
\end{center}

2. Inductive Hypothesis: \\
\indent \indent Assume we have two subtrees of the same depth\\ 
\indent \indent and the number of nodes $n(d-1) = 2^{(d-1)+1} -1$ for each subtree, \\
\indent \indent thus $n(d-1) = 2^d -1$.\\

3. Inductive Step: 
\begin{center}
$n(d) = n(d-1) \times 2 + 1$ \\ 
$n(d) = (2^d-1) \times 2 + 1$\\
$n(d) = 2\times2^d - 2 + 1$\\
$n(d) = 2^{d +1} - 1$ $\square$\\
\end{center}
\indent \indent where the $\times 2$ is derived from the two subtrees, \\
\indent and the $+1$ from the root node.\\

\underline{Problem 2}\\

a) $T(1) = 1$, $T(n) = T(\frac{n}{2}) + 1$ for $n = 2^k$, $k \in \mathbb{N}$\\
\indent \indent $T(n) = T(\frac{n}{2}) + 1$\\
\indent \indent \indent $= (T(\frac{n}{2^2}) + 1) + 1$\\
\indent \indent \indent $=((T(\frac{n}{2^3}) + 1) +1) + 1$\\
\indent \indent \indent repeat this recursion $k$ times...\\
\indent \indent \indent $=T(\frac{n}{2^k}) + k$\\
\indent \indent \indent $=T(\frac{2^k}{2^k}) + k$ as $n = 2^k$\\
\indent \indent \indent $=T(1) + k$\\
\indent \indent \indent $=1 + k$\\
\indent \indent \indent $=1 + log{n}$\\

b) $T(1) = 1$, $T(n) = 2T(\frac{n}{2}) + 1$ for $n = 2^k$, $k \in \mathbb{N}$\\
\indent\indent $T(n) = 2T(\frac{n}{2}) + 1$\\
\indent\indent\indent $= 2(2T(\frac{n}{2^2}) + 1) + 1$\\
\indent\indent\indent $= 2^2T(\frac{n}{2^2}) + 2 + 1$\\
\indent\indent\indent $= 2(2^2T(\frac{n}{2^3}) + 2 + 1) + 1$\\
\indent\indent\indent $= 2^3T(\frac{n}{2^3}) + 4 + 2 + 1$\\
\indent \indent \indent repeat this recursion $k$ times...\\
\indent\indent\indent $= 2^kT(\frac{n}{2^k}) + 2^k - 1$\\
\indent\indent\indent $= 2^kT(1) + 2^k - 1$\\
\indent\indent\indent $= 2 \times 2^k - 1$\\
\indent\indent\indent $= 2n - 1$\\

c) $T(1) = 1$, $T(n) = 2T(\frac{n}{2}) + n$ for $n = 2^k$, $k \in \mathbb{N}$\\
\indent\indent $T(n) = 2T(\frac{n}{2}) + n$\\
\indent\indent\indent $= 2(2T(\frac{n}{2^2}) + \frac{n}{2}) + n$\\
\indent\indent\indent $= 2^2T(\frac{n}{2^2}) + n + n$\\
\indent\indent\indent $= 2(2^2T(\frac{n}{2^3}) + \frac{n}{2} + \frac{n}{2}) + n$\\
\indent\indent\indent $= 2^3T(\frac{n}{2^3}) + n + n + n$\\
\indent\indent\indent $= 2^kT(\frac{n}{2^k}) + kn$\\
\indent\indent\indent $= 2^kT(1) + kn$\\
\indent\indent\indent $= 2^k + kn$\\
\indent\indent\indent $= n + nlog{n}$\\

\underline {Problem 3}\\
\indent a) $\sum_{i=1}^k i(i+1) = k(k+1)(k+2)/3$\\

Base Case P(1):\\
\begin{center}
$1(1+1) = \frac{1(1+1)(1+2)}{3}$\\
$2 = \frac{6}{3}$\\
$2 = 2$\\
\end{center}

Inductive Step:
\begin{center}
$T(k+1) = T(k) + (k+1)((k+1) +1) = T(k) + (k+1)(k+2)$\\
$\frac{(k+1)((k+1)+1)((k+1)+2)}{3} = \frac{k(k+1)(k+2)}{3} + (k+1)(k+2)$\\
$\frac{(k+1)(k+2)(k+3)}{3} = \frac{k(k+1)(k+2)}{3} + (k+1)(k+2)$\\
$\frac{k(k+1)(k+2)}{3} + \frac{3(k+1)(k+2)}{3} = \frac{k(k+1)(k+2)}{3} + (k+1)(k+2)$\\
$\frac{k(k+1)(k+2)}{3} + (k+1)(k+2) = \frac{k(k+1)(k+2)}{3} + (k+1)(k+2)$ $\square$\\
\end{center}

b) $\sum_{i=0}^k i2^i = (k-1)2^{k+1} + 2$\\

Base Case P(0):
\begin{center}
$0(2^0) = (0-1)2^{0+1} + 2$\\
$0 = -2^1 + 2$\\
$0 = 0$\\
\end{center}

Inductive Step:
\begin{center}
$T(k+1) = T(k) + (k+1)2^{k+1}$\\
$(k)2^{(k+1)+1} + 2 = (k-1)2^{k+1} + 2 + (k+1)2^{k+1}$\\
$k2^{k+2} + 2 = k2^{k+1} - 2^{k+1} + 2 + k2^{k+1} + 2^{k+1}$\\
$4k2^k + 2 = 2k2^{k+1} + 2$\\
$4k2^k + 2 = 4k2^k + 2$ $\square$\\
\end{center}

c)$\sum_{i=0}^k \frac{i}{2^i} = 2 - \frac{(k+2)}{2^k}$\\

Base Case P(0):
\begin{center}
$\frac{0}{2^0} = 2 - \frac{(0+2)}{2^0}$\\
$0 = 2 - \frac{2}{1}$\\
$0 = 0$\\
\end{center}

Inductive Step:
\begin{center}
$T(k+1) = T(k) + \frac{(k+1)}{2^{k+1}}$\\
$2 - \frac{((k+1)+2)}{2^{k+1}} = 2 - \frac{(k+2)}{2^k} + \frac{(k+1)}{2^{k+1}}$\\
$\frac{-k-1-2}{2^{k+1}} = \frac{-k-2}{2^k} + \frac{k+1}{2^{k+1}}$\\
$\frac{-k-1}{2^{k+1}} + \frac{-2}{2^{k+1}} = \frac{-k-2}{2^k} + \frac{k+1}{2^{k+1}}$\\
$-2\frac{k+1}{2^{k+1}} + \frac{-2}{2^{k+1}} = \frac{-k-2}{2^k}$\\
$\frac{-k-1}{2^k} + \frac{-1}{2^k} = \frac{-k-2}{2^k}$\\
$\frac{-k-2}{2^k} = \frac{-k-2}{2^k}$ $\square$\\
\end{center}

\underline{Problem 4}\\
\indent Place the following in increasing asymptotic order:\\
\indent \indent $4n, n^2, n\log{n}, n\ln{n}, \log{n}, e^n$\\

First we, drop all constants, so $4n = O(n)$. \\
\indent $O(\log{n}) < O(n) < O(n^2) < O(e^n)$.\\
\indent $O(n\log{n})$ and $O(n\ln{n})$ are both greater than $O(n)$ and less than $O(n^2)$.
 \begin{center}
$ n\log{n} > n\ln{n}$\\
$\log{n} > \ln{n}$\\
 \end{center}

 
Therefore, the correct order is:\\
\indent \indent $ \log{n}, 4n, n\ln{n}, n\log{n}, n^2, e^n$.\\

\underline{Problem 5}\\
\indent Proof by Contradiction:\\
\indent Say we have $T$ and $T*$, both are $MST of G$ and $T \neq T*$\\
\indent Because $\forall e$, $e' \in E$, $w(e) \neq w(e')$ where w(e) is the weight of e,\\ 
\indent all edges can be identified by their unique weight\\
\indent Say $e \in E \in T$, $e' \in E \in T*$, where $e$, $e' \in E \in G$\\
\indent \indent Case 1: $w(e) > w(e')$\\
\indent\indent\indent then $w(T) > w(T*)$, where w(T) is the sum of all weights\\
\indent\indent\indent therefore, T is not an MST of G, which is a contradiction\\
\indent \indent Case 2: $w(e) < w(e')$\\
\indent\indent\indent then $w(T) < w(T*)$, where w(T) is the sum of all weights\\
\indent\indent\indent therefore, T* is not an MST of G, which is a contradiction\\
\indent \indent Case 3: $w(e) = w(e')$\\
\indent\indent\indent then $w(T) = w(T*)$, where w(T) is the sum of all weights\\
\indent\indent\indent therefore, T = T*, which is a contradiction $\square$\\ 

\underline{Problem 6}\\
\indent Given a matrix-representation of a graph, and the following functions: 
\indent \indent Connect(u,v) which runs in $O(1)$\\
\indent \indent Disconnect(u,v) which runs in $O(1)$\\
\indent \indent Adj(u,v) which runs in $O(1)$\\ 
\indent we have developed the following algorithm:\\

HamC(G):\\
\indent \indent Adj(u,v):\\
\indent \indent \indent Disconnect(u,v)\\
\indent \indent \indent HamC(G):\\
\indent \indent \indent \indent FALSE\\
\indent \indent \indent Else:\\
\indent \indent \indent \indent TRUE\\
\indent \indent Else:\\
\indent \indent \indent FALSE\\
\indent Else:\\
\indent  \indent Connect(u,v)\\
\indent \indent HamC(G):\\
\indent \indent \indent TRUE\\
\indent \indent Else:\\
\indent \indent \indent FALSE\\

Due to the use of HamC(G), this algorithm runs in $O(n)$.\\

\end{document}